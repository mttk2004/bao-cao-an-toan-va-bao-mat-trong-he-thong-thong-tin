\chapter{KẾT LUẬN VÀ HƯỚNG PHÁT TRIỂN}

\section{Kết luận}
Đồ án "Phần mềm quản lý mật khẩu an toàn AuraCrypt" đã hoàn thành các mục tiêu đề ra. Tôi đã thành công trong việc thiết kế và triển khai một ứng dụng quản lý mật khẩu an toàn, dễ sử dụng, tích hợp các tính năng bảo mật và tiện ích cần thiết. AuraCrypt cung cấp một giải pháp đáng tin cậy để người dùng bảo vệ thông tin đăng nhập quan trọng của mình.

Những điểm nổi bật của đồ án bao gồm:
\begin{itemize}
    \item Hệ thống mã hóa mạnh mẽ với AES-256-GCM và PBKDF2.
    \item Giao diện người dùng trực quan, thân thiện.
    \item Các tính năng tiện ích như tạo mật khẩu mạnh, quản lý danh mục, sao lưu tự động.
    \item Khả năng đóng gói thành ứng dụng độc lập trên Windows, tự động tạo shortcut.
\end{itemize}
Trong quá trình thực hiện, tôi đã học hỏi được nhiều kiến thức quý báu về mật mã học, kiến trúc phần mềm và quy trình phát triển ứng dụng.

\section{Hạn chế}
Mặc dù đã đạt được nhiều thành công, đồ án vẫn còn một số hạn chế:
\begin{itemize}
    \item \textbf{Chưa hỗ trợ đa nền tảng hoàn chỉnh:} Hiện tại chỉ tập trung tối ưu cho Windows, chưa thử nghiệm đầy đủ trên macOS hoặc Linux với các tính năng đặc thù (ví dụ: tạo shortcut).
    \item \textbf{Chưa tích hợp tự động điền/đăng nhập:} Người dùng vẫn phải sao chép/dán mật khẩu thủ công.
    \item \textbf{Chưa có tính năng đồng bộ hóa:} Dữ liệu chỉ được lưu trữ cục bộ, chưa có cơ chế đồng bộ hóa an toàn giữa các thiết bị.
    \item \textbf{Giao diện cơ bản:} Mặc dù thân thiện, giao diện vẫn có thể được cải thiện để hiện đại và linh hoạt hơn.
\item \textbf{Xử lý ngoại lệ:} Một số trường hợp ngoại lệ hiếm gặp có thể chưa được xử lý tối ưu.
\end{itemize}

\section{Hướng phát triển trong tương lai}
Để cải thiện và mở rộng AuraCrypt, tôi đề xuất các hướng phát triển sau:
\begin{itemize}
    \item \textbf{Tích hợp tính năng tự động điền (Autofill):} Phát triển tiện ích mở rộng trình duyệt hoặc tính năng tự động điền vào các trường đăng nhập trên website.
    \item \textbf{Đồng bộ hóa dữ liệu an toàn:} Nghiên cứu và triển khai cơ chế đồng bộ hóa dữ liệu mã hóa giữa các thiết bị thông qua các dịch vụ đám mây (ví dụ: Google Drive, Dropbox) với bảo mật từ đầu đến cuối (end-to-end encryption).
    \item \textbf{Hỗ trợ đa nền tảng tốt hơn:} Tối ưu hóa cho macOS và Linux, bao gồm cả các tính năng đặc thù của từng hệ điều hành.
    \item \textbf{Xác thực đa yếu tố (MFA):} Tích hợp các phương pháp xác thực thứ cấp (ví dụ: sinh trắc học, TOTP) để tăng cường bảo mật.
    \item \textbf{Cải thiện UI/UX:} Nâng cấp giao diện người dùng, bổ sung thêm các tùy chỉnh và chủ đề (themes) để cá nhân hóa trải nghiệm.
    \item \textbf{Kiểm tra bảo mật chuyên sâu:} Thực hiện các cuộc kiểm tra thâm nhập (penetration testing) và đánh giá lỗ hổng bảo mật chuyên nghiệp.
\end{itemize}

Hy vọng đồ án này sẽ là nền tảng vững chắc cho các nghiên cứu và phát triển tiếp theo trong lĩnh vực quản lý mật khẩu và bảo mật thông tin.
\newpage