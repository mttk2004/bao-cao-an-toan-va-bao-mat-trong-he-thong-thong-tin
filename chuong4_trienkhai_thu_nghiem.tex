\chapter{TRIỂN KHAI VÀ THỬ NGHIỆM}

Chương này trình bày chi tiết quá trình hiện thực hóa ứng dụng AuraCrypt Web, từ việc thiết lập môi trường phát triển đến việc cài đặt các module bảo mật cốt lõi. Đồng thời, chương cũng mô tả các kịch bản thử nghiệm để đánh giá tính đúng đắn và độ an toàn của hệ thống.

\section{Môi trường phát triển}

Để đảm bảo hiệu năng cao và tính bảo mật, ứng dụng được phát triển trên nền tảng công nghệ web hiện đại nhất tính đến thời điểm thực hiện (2025).

\subsection{Công cụ và Thư viện}
\begin{itemize}
    \item \textbf{Ngôn ngữ lập trình:} TypeScript 5.x (đảm bảo an toàn kiểu dữ liệu).
    \item \textbf{Framework:} React 19 (thư viện UI), Vite (công cụ build tốc độ cao).
    \item \textbf{Giao diện:} Tailwind CSS v4 (thiết kế giao diện hiện đại, Responsive).
    \item \textbf{Quản lý trạng thái:} Zustand (quản lý session và khóa mã hóa tạm thời).
    \item \textbf{Backend/Database:} Supabase (PostgreSQL với Row Level Security).
    \item \textbf{Mật mã học:} Web Crypto API (Native Browser API).
\end{itemize}

\subsection{Cấu trúc dự án}
Mã nguồn được tổ chức theo kiến trúc mô-đun hóa, tách biệt rõ ràng giữa giao diện (Components), logic nghiệp vụ (Hooks) và các dịch vụ xử lý nền (Services).

\begin{lstlisting}[language=bash, caption=Cấu trúc thư mục dự án AuraCrypt Web]
src/
|-- components/      # Cac thanh phan giao dien (Dashboard, VaultUnlock...)
|-- hooks/           # Custom Hooks (useVaultData, useVaultAuth...)
|-- services/        # Cac ham xu ly logic nghiep vu
|   |-- cryptoUtils.ts   # Cot loi mat ma hoc (AES-GCM, PBKDF2)
|   |-- healthCheck.ts   # Kiem tra lo lot mat khau (k-Anonymity)
|   |-- shareService.ts  # Logic chia se bao mat
|-- store/           # Quan ly trang thai toan cuc (Zustand)
|-- types.ts         # Dinh nghia kieu du lieu
|-- supabaseClient.ts # Cau hinh ket noi Database
\end{lstlisting}

\section{Hiện thực hóa các chức năng cốt lõi}

\subsection{Module Mật mã học (Crypto Utils)}
Đây là thành phần quan trọng nhất của hệ thống, chịu trách nhiệm thực hiện mọi thao tác mã hóa và giải mã. File \texttt{cryptoUtils.ts} sử dụng trực tiếp \texttt{window.crypto.subtle} để đảm bảo khóa không bao giờ bị rò rỉ qua các thư viện JavaScript không an toàn.

\subsubsection{Dẫn xuất khóa (Key Derivation)}
Hàm \texttt{deriveKeyFromPassword} sử dụng thuật toán PBKDF2 để chuyển đổi mật khẩu người dùng thành khóa AES-256.

\begin{lstlisting}[language=TypeScript, caption=Hàm dẫn xuất khóa sử dụng PBKDF2]
export const deriveKeyFromPassword = async (password: string): Promise<CryptoKey> => {
  const enc = new TextEncoder();
  const keyMaterial = await window.crypto.subtle.importKey(
    "raw", enc.encode(password), "PBKDF2", false, ["deriveBits", "deriveKey"]
  );

  // Su dung 100.000 vong lap de chong Brute-force
  return window.crypto.subtle.deriveKey(
    {
      name: "PBKDF2",
      salt: enc.encode("AURACRYPT_SALT_V1"), // Salt dinh danh ung dung
      iterations: 100000,
      hash: "SHA-256",
    },
    keyMaterial,
    { name: "AES-GCM", length: 256 },
    true, ["encrypt", "decrypt"]
  );
};
\end{lstlisting}

\subsubsection{Mã hóa và Giải mã (Encryption \& Decryption)}
Quá trình mã hóa sử dụng thuật toán AES-GCM. Mỗi lần mã hóa đều sinh ra một IV (Initialization Vector) ngẫu nhiên 12 bytes.

\begin{lstlisting}[language=TypeScript, caption=Hàm mã hóa dữ liệu AES-GCM]
export const encryptData = async (plaintext: string, key: CryptoKey) => {
  // Sinh IV ngau nhien cho moi lan ma hoa
  const iv = window.crypto.getRandomValues(new Uint8Array(12));
  const encoded = new TextEncoder().encode(plaintext);

  const cipherBuffer = await window.crypto.subtle.encrypt(
    { name: "AES-GCM", iv: iv },
    key,
    encoded
  );

  return {
    cipherText: arrayBufferToBase64(cipherBuffer),
    iv: arrayBufferToBase64(iv.buffer),
  };
};
\end{lstlisting}

\subsection{Quy trình Mở khóa Két (Vault Unlock)}
Component \texttt{VaultUnlock.tsx} thực hiện cơ chế xác thực "Verify-before-Unlock". Thay vì tin tưởng mù quáng vào mật khẩu người dùng nhập, hệ thống sẽ tải một bản ghi mã hóa về và thử giải mã nó.

\begin{itemize}
    \item Nếu giải mã thành công: Khóa đúng $\rightarrow$ Cho phép truy cập Dashboard.
    \item Nếu giải mã thất bại (Exception): Khóa sai $\rightarrow$ Báo lỗi và chặn truy cập.
\end{itemize}

\begin{figure}[H]
    \centering
    \includegraphics[width=1\textwidth]{images/ui_unlock.png}
    \caption{Giao diện mở khóa két bảo mật}
    \label{fig:ui_unlock}
\end{figure}

\subsection{Quản lý Dữ liệu và Đồng bộ hóa}
Hook \texttt{useVaultData} chịu trách nhiệm tải dữ liệu thô (đã mã hóa) từ Supabase và thực hiện giải mã phía Client trước khi hiển thị lên giao diện.

Dữ liệu được hiển thị trên Dashboard bao gồm:
\begin{itemize}
    \item \textbf{Metadata (Rõ):} Tên dịch vụ, Tên đăng nhập, Danh mục (để phục vụ tìm kiếm nhanh).
    \item \textbf{Secret Data (Mã hóa):} Mật khẩu, Ghi chú, Thông tin thẻ. Những trường này được hiển thị dưới dạng `••••••' và chỉ hiện rõ khi người dùng yêu cầu (nút Eye icon).
\end{itemize}

\begin{figure}[H]
    \centering
    \includegraphics[width=1\textwidth]{images/ui_dashboard.png}
    \caption{Giao diện Dashboard}
    \label{fig:ui_dashboard}
\end{figure}

\subsection{Tính năng Chia sẻ An toàn (Secure Sharing)}
Tính năng này cho phép người dùng chia sẻ mật khẩu hoặc thông tin nhạy cảm thông qua "Magic Link" (Liên kết thần kỳ) có thời hạn. Điểm đột phá của tính năng này là việc áp dụng cơ chế \textbf{Mã hóa phía Client với Khóa tạm thời (Client-side Encryption with Transient Key)}.

Quy trình hoạt động chi tiết được chia thành hai giai đoạn:

\subsubsection{Quy trình phía Người gửi (Sender)}
\begin{enumerate}
    \item \textbf{Sinh khóa tạm thời:} Khi người dùng chọn chia sẻ một mục, ứng dụng sử dụng \texttt{window.crypto} để sinh ra một khóa AES-256 ngẫu nhiên (Transient Key) ngay tại trình duyệt. Khóa này hoàn toàn độc lập với Mật khẩu chủ và chỉ dùng duy nhất cho lần chia sẻ này.
    \item \textbf{Mã hóa Payload:} Dữ liệu cần chia sẻ (Tên dịch vụ, Tên đăng nhập, Mật khẩu, Ghi chú) được đóng gói thành JSON và mã hóa bằng khóa tạm thời vừa sinh ra.
    \item \textbf{Lưu trữ:} Chỉ có chuỗi mã hóa (Ciphertext), IV và các metadata (thời gian hết hạn, số lượt xem) được gửi lên Server (Supabase). Server trả về một định danh duy nhất (UUID) cho bản ghi này.
    \item \textbf{Tạo liên kết:} Ứng dụng tạo ra đường dẫn chia sẻ có định dạng:
    \begin{center}
        \texttt{https://domain.app/\#share?id=<UUID>\&key=<Base64Key>}
    \end{center}
    \textbf{Cơ chế bảo mật:} Do khóa giải mã được đặt sau dấu thăng (\# - Hash Fragment), theo tiêu chuẩn HTTP, trình duyệt sẽ \textbf{không bao giờ} gửi phần này lên Server khi thực hiện HTTP Request. Do đó, Server nắm giữ dữ liệu nhưng không bao giờ biết được khóa giải mã.
\end{enumerate}

\subsubsection{Quy trình phía Người nhận (Recipient)}
\begin{enumerate}
    \item Khi người nhận truy cập liên kết, ứng dụng React sẽ được tải về trình duyệt.
    \item Ứng dụng đọc URL hiện tại, trích xuất tham số \texttt{id} và \texttt{key} từ phần Hash.
    \item Ứng dụng gọi API lên Supabase để tải gói dữ liệu mã hóa dựa trên \texttt{id}.
    \item Nếu liên kết chưa hết hạn và còn lượt xem, ứng dụng sử dụng \texttt{key} lấy từ URL để giải mã dữ liệu ngay tại trình duyệt của người nhận và hiển thị thông tin.
\end{enumerate}

\begin{lstlisting}[language=TypeScript, caption=Logic tạo URL chia sẻ an toàn]
// 1. Tao khoa tam thoi (Transient Key)
const transientKey = await generateTransientKey();

// 2. Ma hoa du lieu bang khoa tam thoi
const { cipherText, iv } = await encryptData(jsonData, transientKey);

// 3. Upload Ciphertext lên Server (Server khong biet Key)
const { data } = await supabase.from('shared_entries').insert({
    encrypted_data: cipherText,
    iv: iv,
    // ... metadata khac
}).select().single();

// 4. Xuat Key ra chuoi va ghep vao URL Hash
const keyJson = await exportKeyToString(transientKey);
// Hash Fragment (#) dam bao key khong bao gio duoc gui len server
const shareUrl = `${window.location.origin}/#share?id=${data.id}&key=${btoa(keyJson)}`;
\end{lstlisting}

\begin{figure}[H]
    \centering
    \includegraphics[width=1\textwidth]{images/ui_share_modal.png}
    \caption{Giao diện cấu hình chia sẻ (Thời hạn, Số lượt xem)}
    \label{fig:ui_share_modal}
\end{figure}

\begin{figure}[H]
    \centering
    \includegraphics[width=1\textwidth]{images/ui_share_modal_2.png}
    \caption{Liên kết chia sẻ được tạo ra (Khóa nằm trong phần Hash)}
    \label{fig:ui_share_modal_2}
\end{figure}

\begin{figure}[H]
    \centering
    \includegraphics[width=1\textwidth]{images/ui_share_view.png}
    \caption{Giao diện người nhận: Yêu cầu xác nhận mở khóa}
    \label{fig:ui_share_view}
\end{figure}

\begin{figure}[H]
    \centering
    \includegraphics[width=1\textwidth]{images/ui_share_view_2.png}
    \caption{Giao diện người nhận: Dữ liệu sau khi được giải mã phía Client}
    \label{fig:ui_share_view_2}
\end{figure}

\section{Thử nghiệm và Đánh giá}

\subsection{Kịch bản thử nghiệm chức năng}
Hệ thống đã được kiểm thử với các kịch bản sử dụng thực tế:

\begin{longtable}{|p{1cm}|p{4cm}|p{5cm}|p{3cm}|}
    \hline
    \textbf{STT} & \textbf{Chức năng} & \textbf{Quy trình thực hiện} & \textbf{Kết quả} \\
    \hline
    1 & Đăng ký/Đăng nhập & Tạo tài khoản mới, đăng nhập, thiết lập Mật khẩu chủ. & Thành công. \\
    \hline
    2 & Mã hóa dữ liệu & Thêm mới một mục nhập (Entry). Kiểm tra trong Database Supabase. & Dữ liệu trên DB là chuỗi mã hóa Base64 vô nghĩa. \\
    \hline
    3 & Mở khóa (Đúng Pass) & Đăng nhập lại, nhập đúng Mật khẩu chủ. & Dữ liệu được giải mã và hiển thị chính xác. \\
    \hline
    4 & Mở khóa (Sai Pass) & Đăng nhập lại, nhập sai Mật khẩu chủ. & Hệ thống báo lỗi, không hiển thị dữ liệu. \\
    \hline
    5 & Chia sẻ liên kết & Tạo link chia sẻ, mở link trên trình duyệt ẩn danh. & Dữ liệu được giải mã thành công. Server không log lại key. \\
    \hline
\end{longtable}

\subsection{Kiểm thử bảo mật (Security Audit)}
Để xác minh tính chất Zero-Knowledge, chúng tôi đã thực hiện kiểm tra luồng dữ liệu mạng (Network Traffic Analysis):

\begin{itemize}
    \item \textbf{Tại Client:} Khi người dùng bấm "Lưu", payload gửi đi trong tab Network (DevTools) hoàn toàn là các chuỗi mã hóa (\texttt{encrypted\_password}, \texttt{iv}). Không có bất kỳ trường nào chứa mật khẩu gốc (plaintext).
    \item \textbf{Tại Server:} Truy cập Dashboard quản trị của Supabase, kiểm tra bảng \texttt{entries}. Kết quả cho thấy các cột nhạy cảm đều chứa dữ liệu rác (ciphertext), chứng minh server không đọc được nội dung.
\end{itemize}

\begin{figure}[H]
    \centering
    \includegraphics[width=1\textwidth]{images/network.jpg}
    \caption{Payload gửi đi trong tab Network}
    \label{fig:network}
\end{figure}

\begin{figure}[H]
    \centering
    \includegraphics[width=1\textwidth]{images/supabase.png}
    \caption{Dữ liệu bảng entries trong Supabase}
    \label{fig:supabase}
\end{figure}

\subsection{Đánh giá hiệu năng}
\begin{itemize}
    \item \textbf{Tốc độ mã hóa}: Với Web Crypto API, việc mã hóa/giải mã 100 bản ghi diễn ra trong thời gian dưới 200ms, gần như tức thì đối với trải nghiệm người dùng.
    \item \textbf{Trải nghiệm người dùng}: Giao diện React phản hồi mượt mà. Các thao tác chuyển trang không cần tải lại (SPA). Chế độ Dark Mode hoạt động tốt, giảm mỏi mắt khi làm việc lâu.
\end{itemize}

\section{Triển khai trên môi trường thực tế (Deployment)}

Để kiểm chứng khả năng hoạt động của ứng dụng trong môi trường Internet thực tế, tôi đã tiến hành đóng gói và triển khai AuraCrypt Web lên hạ tầng đám mây.

\subsection{Quy trình đóng gói (Build Process)}
Ứng dụng React được biên dịch từ mã nguồn TypeScript sang mã JavaScript tĩnh (Static Assets) tối ưu hóa cho môi trường production thông qua công cụ Vite.

\begin{lstlisting}[language=bash, caption=Lệnh đóng gói ứng dụng]
# Cai dat cac phu thuoc
npm install

# Bien dich ung dung (tao thu muc /dist)
npm run build
\end{lstlisting}

Quá trình này thực hiện các tác vụ:
\begin{itemize}
    \item Kiểm tra cú pháp và kiểu dữ liệu (Type checking).
    \item Nén mã nguồn (Minification) để giảm dung lượng tải.
    \item Tạo các file Hash cho tài nguyên để tối ưu hóa bộ nhớ đệm (Browser Caching).
\end{itemize}

\subsection{Hạ tầng vận hành}
Ứng dụng được triển khai trên nền tảng **Netlify** – một dịch vụ Hosting chuyên dụng cho các ứng dụng web hiện đại (Jamstack).
\begin{itemize}
    \item \textbf{Địa chỉ truy cập:} \url{https://auracrypt.netlify.app/}
    \item \textbf{Môi trường:} HTTPS (TLS 1.3) được kích hoạt mặc định để đảm bảo kênh truyền an toàn, yếu tố bắt buộc để Web Crypto API hoạt động.
    \item \textbf{CI/CD:} Hệ thống được cấu hình để tự động cập nhật mỗi khi có thay đổi mới trên kho mã nguồn GitHub.
\end{itemize}

% --- Bắt đầu đoạn mã chèn QR ---
\begin{figure}[H]
    \centering
    % Tạo mã QR, độ cao 4cm
    % Lệnh \href bao bên ngoài để khi xem PDF trên máy tính có thể click vào QR để mở link luôn
    \href{https://auracrypt.netlify.app/}{
        \qrcode[height=4cm]{https://auracrypt.netlify.app/}
    }
    
    \vspace{0.5cm} % Khoảng cách giữa QR và chú thích
    
    \caption{Quét mã để trải nghiệm AuraCrypt Web ngay trên thiết bị di động}
    \label{fig:qr_code}
    
    % Hiển thị thêm link text bên dưới cho rõ ràng
    \vspace{0.2cm}
    \small{\url{https://auracrypt.netlify.app/}}
\end{figure}
% --- Kết thúc đoạn mã chèn QR ---

\begin{figure}[H]
    \centering
    \includegraphics[width=1\textwidth]{images/deployed_app.png}
    \caption{Ứng dụng AuraCrypt hoạt động trên môi trường Internet thực tế}
    \label{fig:deployed_app}
\end{figure}

\newpage
