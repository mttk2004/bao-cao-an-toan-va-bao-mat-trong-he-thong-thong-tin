\chapter{TRIỂN KHAI VÀ THỬ NGHIỆM}

Chương này trình bày chi tiết quá trình hiện thực hóa ứng dụng AuraCrypt Web, từ việc thiết lập môi trường phát triển đến việc cài đặt các module bảo mật cốt lõi. Đồng thời, chương cũng mô tả các kịch bản thử nghiệm để đánh giá tính đúng đắn và độ an toàn của hệ thống.

\section{Môi trường phát triển}

Để đảm bảo hiệu năng cao và tính bảo mật, ứng dụng được phát triển trên nền tảng công nghệ web hiện đại nhất tính đến thời điểm thực hiện (2025).

\subsection{Công cụ và Thư viện}
\begin{itemize}
    \item \textbf{Ngôn ngữ lập trình:} TypeScript 5.x (đảm bảo an toàn kiểu dữ liệu).
    \item \textbf{Framework:} React 19 (thư viện UI), Vite (công cụ build tốc độ cao).
    \item \textbf{Giao diện:} Tailwind CSS v4 (thiết kế giao diện hiện đại, Responsive).
    \item \textbf{Quản lý trạng thái:} Zustand (quản lý session và khóa mã hóa tạm thời).
    \item \textbf{Backend/Database:} Supabase (PostgreSQL với Row Level Security).
    \item \textbf{Mật mã học:} Web Crypto API (Native Browser API).
\end{itemize}

\subsection{Cấu trúc dự án}
Mã nguồn được tổ chức theo kiến trúc mô-đun hóa, tách biệt rõ ràng giữa giao diện (Components), logic nghiệp vụ (Hooks) và các dịch vụ xử lý nền (Services).

\begin{lstlisting}[language=bash, caption=Cấu trúc thư mục dự án AuraCrypt Web]
src/
|-- components/      # Các thành phần giao diện (Dashboard, VaultUnlock...)
|-- hooks/           # Custom Hooks (useVaultData, useVaultAuth...)
|-- services/        # Các hàm xử lý logic nghiệp vụ
|   |-- cryptoUtils.ts   # Cốt lõi mật mã học (AES-GCM, PBKDF2)
|   |-- healthCheck.ts   # Kiểm tra lộ lọt mật khẩu (k-Anonymity)
|   |-- shareService.ts  # Logic chia sẻ bảo mật
|-- store/           # Quản lý trạng thái toàn cục (Zustand)
|-- types.ts         # Định nghĩa kiểu dữ liệu
|-- supabaseClient.ts # Cấu hình kết nối Database
\end{lstlisting}

\section{Hiện thực hóa các chức năng cốt lõi}

\subsection{Module Mật mã học (Crypto Utils)}
Đây là thành phần quan trọng nhất của hệ thống, chịu trách nhiệm thực hiện mọi thao tác mã hóa và giải mã. File \texttt{cryptoUtils.ts} sử dụng trực tiếp \texttt{window.crypto.subtle} để đảm bảo khóa không bao giờ bị rò rỉ qua các thư viện JavaScript không an toàn.

\subsubsection{Dẫn xuất khóa (Key Derivation)}
Hàm \texttt{deriveKeyFromPassword} sử dụng thuật toán PBKDF2 để chuyển đổi mật khẩu người dùng thành khóa AES-256.

\begin{lstlisting}[language=TypeScript, caption=Hàm dẫn xuất khóa sử dụng PBKDF2]
export const deriveKeyFromPassword = async (password: string): Promise<CryptoKey> => {
  const enc = new TextEncoder();
  const keyMaterial = await window.crypto.subtle.importKey(
    "raw", enc.encode(password), "PBKDF2", false, ["deriveBits", "deriveKey"]
  );

  // Sử dụng 100.000 vòng lặp để chống Brute-force
  return window.crypto.subtle.deriveKey(
    {
      name: "PBKDF2",
      salt: enc.encode("AURACRYPT_SALT_V1"), // Salt định danh ứng dụng
      iterations: 100000,
      hash: "SHA-256",
    },
    keyMaterial,
    { name: "AES-GCM", length: 256 },
    true, ["encrypt", "decrypt"]
  );
};
\end{lstlisting}

\subsubsection{Mã hóa và Giải mã (Encryption \& Decryption)}
Quá trình mã hóa sử dụng thuật toán AES-GCM. Mỗi lần mã hóa đều sinh ra một IV (Initialization Vector) ngẫu nhiên 12 bytes.

\begin{lstlisting}[language=TypeScript, caption=Hàm mã hóa dữ liệu AES-GCM]
export const encryptData = async (plaintext: string, key: CryptoKey) => {
  // Sinh IV ngẫu nhiên cho mỗi lần mã hóa
  const iv = window.crypto.getRandomValues(new Uint8Array(12));
  const encoded = new TextEncoder().encode(plaintext);

  const cipherBuffer = await window.crypto.subtle.encrypt(
    { name: "AES-GCM", iv: iv },
    key,
    encoded
  );

  return {
    cipherText: arrayBufferToBase64(cipherBuffer),
    iv: arrayBufferToBase64(iv.buffer),
  };
};
\end{lstlisting}

\subsection{Quy trình Mở khóa Két (Vault Unlock)}
Component \texttt{VaultUnlock.tsx} thực hiện cơ chế xác thực "Verify-before-Unlock". Thay vì tin tưởng mù quáng vào mật khẩu người dùng nhập, hệ thống sẽ tải một bản ghi mã hóa về và thử giải mã nó.

\begin{itemize}
    \item Nếu giải mã thành công: Khóa đúng $\rightarrow$ Cho phép truy cập Dashboard.
    \item Nếu giải mã thất bại (Exception): Khóa sai $\rightarrow$ Báo lỗi và chặn truy cập.
\end{itemize}

% // TODO: Chèn hình ảnh giao diện Màn hình Mở khóa (VaultUnlock) và Màn hình Setup (khi user mới đăng ký)
\begin{figure}[H]
    \centering
    % \includegraphics[width=0.7\textwidth]{images/vault_unlock_ui.png}
    \caption{Giao diện mở khóa két bảo mật}
    \label{fig:unlock_ui}
\end{figure}

\subsection{Quản lý Dữ liệu và Đồng bộ hóa}
Hook \texttt{useVaultData} chịu trách nhiệm tải dữ liệu thô (đã mã hóa) từ Supabase và thực hiện giải mã phía Client trước khi hiển thị lên giao diện.

Dữ liệu được hiển thị trên Dashboard bao gồm:
\begin{itemize}
    \item \textbf{Metadata (Rõ):} Tên dịch vụ, Tên đăng nhập, Danh mục (để phục vụ tìm kiếm nhanh).
    \item \textbf{Secret Data (Mã hóa):} Mật khẩu, Ghi chú, Thông tin thẻ. Những trường này được hiển thị dưới dạng `••••••' và chỉ hiện rõ khi người dùng yêu cầu (nút Eye icon).
\end{itemize}

% // TODO: Chèn hình ảnh giao diện Dashboard chính với danh sách mật khẩu

\subsection{Tính năng Chia sẻ An toàn (Secure Sharing)}
Tính năng này cho phép chia sẻ mật khẩu qua "Magic Link". Điểm đặc biệt là khóa giải mã được nhúng vào phần Hash của URL (\texttt{\#...}) nên không bao giờ được gửi lên Server.

\begin{lstlisting}[language=TypeScript, caption=Cơ chế tạo URL chia sẻ an toàn]
// Tạo khóa tạm thời (Transient Key)
const transientKey = await generateTransientKey();
const { cipherText, iv } = await encryptData(jsonData, transientKey);

// Upload Ciphertext lên Server, nhưng giữ Key lại Client
// ... (Code upload Supabase) ...

// Xuất Key ra chuỗi và ghép vào URL Hash
const keyJson = await exportKeyToString(transientKey);
const shareUrl = `${window.location.origin}/#share?id=${data.id}&key=${btoa(keyJson)}`;
\end{lstlisting}

% // TODO: Chèn hình ảnh giao diện Share Modal và Giao diện người nhận (Share View)

\section{Thử nghiệm và Đánh giá}

\subsection{Kịch bản thử nghiệm chức năng}
Hệ thống đã được kiểm thử với các kịch bản sử dụng thực tế:

\begin{longtable}{|p{1cm}|p{4cm}|p{5cm}|p{3cm}|}
    \hline
    \textbf{STT} & \textbf{Chức năng} & \textbf{Quy trình thực hiện} & \textbf{Kết quả} \\
    \hline
    1 & Đăng ký/Đăng nhập & Tạo tài khoản mới, đăng nhập, thiết lập Mật khẩu chủ. & Thành công. \\
    \hline
    2 & Mã hóa dữ liệu & Thêm mới một mục nhập (Entry). Kiểm tra trong Database Supabase. & Dữ liệu trên DB là chuỗi mã hóa Base64 vô nghĩa. \\
    \hline
    3 & Mở khóa (Đúng Pass) & Đăng nhập lại, nhập đúng Mật khẩu chủ. & Dữ liệu được giải mã và hiển thị chính xác. \\
    \hline
    4 & Mở khóa (Sai Pass) & Đăng nhập lại, nhập sai Mật khẩu chủ. & Hệ thống báo lỗi, không hiển thị dữ liệu. \\
    \hline
    5 & Chia sẻ liên kết & Tạo link chia sẻ, mở link trên trình duyệt ẩn danh. & Dữ liệu được giải mã thành công. Server không log lại key. \\
    \hline
\end{longtable}

\subsection{Kiểm thử bảo mật (Security Audit)}
Để xác minh tính chất Zero-Knowledge, chúng tôi đã thực hiện kiểm tra luồng dữ liệu mạng (Network Traffic Analysis):

\begin{itemize}
    \item \textbf{Tại Client:} Khi người dùng bấm "Lưu", payload gửi đi trong tab Network (DevTools) hoàn toàn là các chuỗi mã hóa (\texttt{encrypted\_password}, \texttt{iv}). Không có bất kỳ trường nào chứa mật khẩu gốc (plaintext).
    \item \textbf{Tại Server:} Truy cập Dashboard quản trị của Supabase, kiểm tra bảng \texttt{entries}. Kết quả cho thấy các cột nhạy cảm đều chứa dữ liệu rác (ciphertext), chứng minh server không đọc được nội dung.
\end{itemize}

% // TODO: Chèn hình ảnh chụp Database Supabase (để chứng minh dữ liệu bị mã hóa) và hình ảnh Network Tab

\subsection{Đánh giá hiệu năng}
\begin{itemize}
    \item **Tốc độ mã hóa:** Với Web Crypto API, việc mã hóa/giải mã 100 bản ghi diễn ra trong thời gian dưới 200ms, gần như tức thì đối với trải nghiệm người dùng.
    \item **Trải nghiệm người dùng:** Giao diện React phản hồi mượt mà. Các thao tác chuyển trang không cần tải lại (SPA). Chế độ Dark Mode hoạt động tốt, giảm mỏi mắt khi làm việc lâu.
\end{itemize}

\newpage
