\documentclass[11pt, a4paper]{report}

% --- CÀI ĐẶT UNIVERSAL PREAMBLE ---
% Gói này rất quan trọng để hỗ trợ tiếng Việt
\usepackage{fontspec} % Cho phép chọn font
\usepackage{polyglossia}
\setdefaultlanguage{vietnamese}

% Cấu hình lề giấy
\usepackage[a4paper, top=2.5cm, bottom=2.5cm, left=2.5cm, right=2.5cm]{geometry}

\usepackage{array}
\usepackage{longtable}
% \usepackage{geometry} % Đã định nghĩa ở trên, có thể bỏ dòng này
% \geometry{a4paper, margin=1in} % Đã định nghĩa ở trên, có thể bỏ dòng này
\usepackage{booktabs} % Cho toprule, midrule, bottomrule
\usepackage{amsmath} % Gói toán học cơ bản, đôi khi cần cho các ký tự
\usepackage{amsfonts}
\usepackage{amssymb}
\usepackage{enumitem} % Để tùy chỉnh list (gạch đầu dòng)
\usepackage{float}    % Để tùy chỉnh vị trí hình ảnh và bảng

% Chọn font Times New Roman cho chữ chính và JetBrains Mono cho code
\setmainfont{Times New Roman}
\setmonofont{JetBrains Mono}[Scale=0.9]

% Định nghĩa font riêng cho listings
\newfontfamily\listingsfont{JetBrains Mono}[Scale=0.9]

% Sửa lỗi hiển thị gạch đầu dòng (list bullets)
% \setlist[itemize]{label=-} % Lỗi này thường không xảy ra với polyglossia và fontspec, có thể bỏ nếu không cần thiết.

% --- CÁC GÓI BỔ SUNG CHO BÁO CÁO NÀY ---
\usepackage{graphicx}     % Để chèn hình ảnh (biểu đồ)
% \usepackage{booktabs}     % Đã khai báo ở trên
\usepackage{listings}     % Để chèn code Java (bạn sẽ dùng Python)
\usepackage{xcolor}       % Để định nghĩa màu cho code
\usepackage[unicode=true]{hyperref} % Để hỗ trợ tiếng Việt trong PDF bookmarks, nên đặt cuối cùng

% Cấu hình màu sắc cho code listing
\definecolor{codegreen}{rgb}{0,0.6,0}
\definecolor{codegray}{rgb}{0.5,0.5,0.5}
\definecolor{codepurple}{rgb}{0.58,0,0.82}
\definecolor{backcolour}{rgb}{0.98,0.98,0.98}

% Cấu hình style cho code Python với XeLaTeX
\lstdefinestyle{mystyle}{
    backgroundcolor=\color{backcolour},
    commentstyle=\color{codegreen},
    keywordstyle=\color{magenta}, % Hoặc màu khác cho Python keywords
    numberstyle=\tiny\color{codegray},
    stringstyle=\color{codepurple},
    basicstyle=\listingsfont\footnotesize,
    breakatwhitespace=false,
    breaklines=true,
    captionpos=b,
    keepspaces=true,
    numbers=left,
    numbersep=5pt,
    showspaces=false,
    showstringspaces=false,
    showtabs=false,
    tabsize=2,
    language=Python % <-- CHỈ ĐỊNH NGÔN NGỮ LẬP TRÌNH LÀ Python
}

% Cấu hình chung cho listings
\lstset{style=mystyle}

% --- BẮT ĐẦU TÀI LIỆU ---
\begin{document}

% --- TRANG BÌA ---
\begin{titlepage}
    \centering
    \sffamily

    {\Large ỦY BAN NHÂN DÂN THÀNH PHỐ HỒ CHÍ MINH\par}
    {\LARGE \textbf{TRƯỜNG ĐẠI HỌC SÀI GÒN}\par}
    {\Large \textbf{KHOA CÔNG NGHỆ THÔNG TIN}\par}
    \vspace{1cm}

    \includegraphics[width=0.3\textwidth]{images/logo_sgu.png}

    \vfill

    {\Huge \textbf{BÁO CÁO}}\par
    \vspace{0.5cm}
    {\Large HỌC PHẦN: AN TOÀN VÀ BẢO MẬT DỮ LIỆU TRONG HỆ THỐNG THÔNG TIN}\par
    
    \vspace{1cm}
    {\Huge \textbf{PHẦN MỀM QUẢN LÝ MẬT KHẨU AN TOÀN}}\par

    \vfill

    \begin{flushright}
    \large
    \begin{tabular}{ll}
        \textbf{Giảng viên:} & Trương Tấn Khoa \\
        \textbf{Nhóm thực hiện:} & 46 \\
    \end{tabular}
    \end{flushright}

    \vspace{0.5cm}

    \textbf{\large Danh sách thành viên:}\par
    \large Mai Trần Tuấn Kiệt – 3122560038\par

    \vfill

    {\large Thành phố Hồ Chí Minh, tháng 11 năm 2025\par}
\end{titlepage}

% --- ĐÁNH SỐ TRANG RÔ-MA CHO CÁC TRANG MỞ ĐẦU ---
\pagenumbering{roman}
\chapter*{LỜI CẢM ƠN}
\addcontentsline{toc}{chapter}{LỜI CẢM ƠN}

Để hoàn thành đồ án môn học "An toàn và Bảo mật Dữ liệu trong Hệ thống Thông tin" với đề tài \textbf{"Xây dựng ứng dụng quản lý mật khẩu Web AuraCrypt với kiến trúc Zero-Knowledge"}, tôi đã nhận được sự hỗ trợ, hướng dẫn và động viên rất lớn từ quý Thầy Cô, gia đình và bạn bè.

\vspace{0.5cm}

Lời đầu tiên, tôi xin bày tỏ lòng biết ơn sâu sắc nhất đến \textbf{Thầy Trương Tấn Khoa}. Thầy đã tận tình truyền đạt những kiến thức nền tảng quý báu về an toàn thông tin, định hướng đề tài và đưa ra những lời khuyên thiết thực giúp tôi giải quyết các vấn đề kỹ thuật phức tạp trong quá trình nghiên cứu kiến trúc Zero-Knowledge. Sự chỉ dẫn của Thầy chính là kim chỉ nam giúp tôi hoàn thiện sản phẩm này.

\vspace{0.5cm}

Tôi xin chân thành cảm ơn Ban Giám hiệu \textbf{Trường Đại học Sài Gòn} cùng quý Thầy Cô \textbf{Khoa Công nghệ Thông tin} đã tạo môi trường học tập thuận lợi, cung cấp cơ sở vật chất và tài liệu cần thiết để sinh viên chúng tôi có điều kiện tốt nhất để nghiên cứu và phát triển kỹ năng chuyên môn.

\vspace{0.5cm}

Cuối cùng, tôi xin gửi lời tri ân đến gia đình và bạn bè đã luôn ở bên động viên, chia sẻ và ủng hộ tinh thần cho tôi trong suốt thời gian thực hiện đồ án.

\vspace{0.5cm}

Mặc dù đã nỗ lực hết sức để hoàn thiện đồ án, nhưng do giới hạn về thời gian và kiến thức, báo cáo khó tránh khỏi những thiếu sót. Tôi rất mong nhận được những ý kiến đóng góp quý báu từ Thầy và các bạn để có thể hoàn thiện và phát triển đề tài này tốt hơn trong tương lai.

\vspace{0.5cm}

Tôi xin chân thành cảm ơn!

\vspace{1cm}
\begin{flushright}
\textbf{Sinh viên thực hiện}\par
Mai Trần Tuấn Kiệt\par
\end{flushright}
\newpage % Lời cảm ơn (tùy chọn)
\chapter*{TÓM TẮT}
\addcontentsline{toc}{chapter}{TÓM TẮT} % Thêm vào mục lục thủ công

Đồ án này tập trung vào việc nghiên cứu, thiết kế và triển khai một phần mềm quản lý mật khẩu an toàn mang tên AuraCrypt. Với sự gia tăng của các dịch vụ trực tuyến, việc quản lý và bảo mật một lượng lớn mật khẩu đã trở thành thách thức lớn đối với người dùng. AuraCrypt được phát triển nhằm giải quyết vấn đề này, cung cấp một giải pháp hiệu quả để lưu trữ, mã hóa và quản lý thông tin đăng nhập một cách an toàn trên nhiều nền tảng.

Phần mềm sử dụng thuật toán mã hóa đối xứng tiên tiến AES-256-GCM để bảo vệ toàn bộ dữ liệu người dùng, kết hợp với PBKDF2 để tăng cường độ an toàn cho mật khẩu chủ. Ngoài ra, AuraCrypt tích hợp các tính năng tiện ích như tạo mật khẩu mạnh, quản lý danh mục thông minh, sao lưu tự động và khả năng xuất/nhập dữ liệu. Giao diện người dùng được thiết kế trực quan, thân thiện, giúp người dùng dễ dàng thao tác.

Kết quả của đồ án là một ứng dụng quản lý mật khẩu độc lập trên Windows, đảm bảo tính bảo mật cao và mang lại trải nghiệm người dùng liền mạch, góp phần nâng cao ý thức và hiệu quả bảo mật thông tin cá nhân.

\vspace{1cm}
\begin{flushright}
\textbf{Sinh viên thực hiện}\par
Mai Trần Tuấn Kiệt\par
\end{flushright}
\newpage     % Tóm tắt (Abstract)
% \input{muc_luc}     % Mục lục (nếu muốn một file riêng)
\tableofcontents % Nếu bạn muốn nó tự động tạo, hãy để đây.

\listoffigures % Danh sách hình ảnh (nếu có nhiều hình)
\listoftables  % Danh sách bảng biểu (nếu có nhiều bảng)
\newpage

% --- ĐÁNH SỐ TRANG SỐ CHO NỘI DUNG CHÍNH ---
\pagenumbering{arabic}
\setcounter{page}{1} % Bắt đầu lại số trang từ 1

% --- NẠP CÁC CHƯƠNG TỪ CÁC FILE RIÊNG ---
\chapter{GIỚI THIỆU}
\section{Đặt vấn đề}
Trong thời đại số hóa hiện nay, mỗi cá nhân và tổ chức đều phải quản lý một lượng lớn tài khoản trực tuyến với các mật khẩu khác nhau. Việc sử dụng mật khẩu yếu, trùng lặp hoặc ghi nhớ chúng một cách thủ công không chỉ gây bất tiện mà còn tiềm ẩn nguy cơ bảo mật nghiêm trọng. Các cuộc tấn công mạng, rò rỉ dữ liệu ngày càng phức tạp đã đặt ra yêu cầu cấp thiết về một giải pháp quản lý mật khẩu an toàn và hiệu quả.

\section{Mục tiêu đồ án}
Đồ án "Phần mềm quản lý mật khẩu an toàn AuraCrypt" nhằm đạt được các mục tiêu sau:
\begin{itemize}
    \item Thiết kế và triển khai một ứng dụng quản lý mật khẩu với khả năng mã hóa dữ liệu mạnh mẽ, đảm bảo tính bảo mật và toàn vẹn của thông tin.
    \item Phát triển một giao diện người dùng trực quan, thân thiện, dễ sử dụng cho người dùng cuối.
    \item Tích hợp các tính năng hỗ trợ như tạo mật khẩu mạnh, quản lý danh mục, tìm kiếm nhanh và sao lưu dữ liệu tự động.
    \item Đảm bảo khả năng hoạt động trên nền tảng Windows dưới dạng một ứng dụng độc lập, không yêu cầu cài đặt thêm các thư viện Python.
\end{itemize}

\section{Tổng quan về AuraCrypt}
AuraCrypt là một phần mềm quản lý mật khẩu được phát triển bằng Python sử dụng framework Flet. Phần mềm này cung cấp một kho lưu trữ an toàn cho các thông tin đăng nhập, bảo vệ chúng bằng thuật toán mã hóa AES-256-GCM. Người dùng có thể dễ dàng thêm, sửa, xóa, tìm kiếm và phân loại các mục nhập mật khẩu. AuraCrypt cũng bao gồm các tính năng như tạo mật khẩu ngẫu nhiên, tự động sao lưu dữ liệu và một quy trình thiết lập ban đầu thân thiện với người dùng.

\section{Cấu trúc báo cáo}
Báo cáo này được cấu trúc thành 5 chương chính:
\begin{itemize}
    \item \textbf{Chương 1: Giới thiệu} - Trình bày bối cảnh, đặt vấn đề, mục tiêu và tổng quan về đồ án.
    \item \textbf{Chương 2: Cơ sở lý thuyết} - Tổng hợp các kiến thức nền tảng về mật mã học, thuật toán mã hóa, quản lý mật khẩu và các công nghệ liên quan.
    \item \textbf{Chương 3: Phân tích và thiết kế} - Chi tiết hóa các yêu cầu, phân tích hệ thống, kiến trúc phần mềm và thiết kế giao diện người dùng.
    \item \textbf{Chương 4: Triển khai và thử nghiệm} - Mô tả quá trình hiện thực hóa các tính năng, công nghệ sử dụng và kết quả thử nghiệm.
    \item \textbf{Chương 5: Kết luận và hướng phát triển} - Đánh giá tổng thể đồ án, những hạn chế và các định hướng phát triển trong tương lai.
\end{itemize}
\newpage
\chapter{CƠ SỞ LÝ THUYẾT}

Chương này cung cấp một cái nhìn toàn diện về các nền tảng lý thuyết được sử dụng trong AuraCrypt. Thay vì chỉ liệt kê các thông số kỹ thuật khô khan, chúng tôi sẽ đi sâu vào lịch sử hình thành, lý do ra đời và cơ chế hoạt động của từng công nghệ dưới góc độ dễ hiểu nhất, nhằm làm rõ vai trò thiết yếu của chúng trong việc bảo vệ dữ liệu người dùng.

\section{Kiến trúc Zero-Knowledge (Không kiến thức)}

\subsection{Khái niệm và sự ra đời}
Trong mô hình bảo mật truyền thống (Server-side Encryption), khi người dùng gửi một dữ liệu lên đám mây, máy chủ sẽ nắm giữ cả dữ liệu và chìa khóa để mở nó. Điều này giống như việc khách hàng gửi tiền vào ngân hàng và giao luôn chìa khóa két sắt cho nhân viên ngân hàng. Khách hàng phải đặt niềm tin tuyệt đối rằng nhân viên đó sẽ không tò mò mở két, hoặc ngân hàng sẽ không bị cướp. Tuy nhiên, lịch sử đã chứng minh "niềm tin" là mắt xích yếu nhất trong an ninh mạng qua hàng loạt vụ lộ lọt dữ liệu lớn (như vụ rò rỉ của Yahoo, LinkedIn hay Dropbox).

\vspace{0.5cm}

Kiến trúc \textbf{Zero-Knowledge} (hay còn gọi là Client-side Encryption) ra đời để giải quyết vấn đề "niềm tin" này. Thuật ngữ này ám chỉ việc nhà cung cấp dịch vụ (Service Provider) có "kiến thức bằng không" (zero knowledge) về dữ liệu thực tế của người dùng.

\subsection{Cơ chế hoạt động đơn giản hóa}
Hãy tưởng tượng quy trình hoạt động của AuraCrypt như sau:
\begin{enumerate}
    \item Trước khi rời khỏi máy tính của người dùng, dữ liệu được bỏ vào một chiếc hộp sắt và khóa lại bằng một chìa khóa đặc biệt.
    \item Chiếc chìa khóa này (được tạo ra từ Mật khẩu chủ - Master Password) nằm trong túi của người dùng và \textbf{không bao giờ} được gửi đi đâu cả.
    \item Chiếc hộp sắt (đã khóa) được gửi lên máy chủ (Supabase) để lưu trữ.
    \item Khi cần sử dụng, máy chủ trả lại chiếc hộp sắt. Người dùng dùng chìa khóa trong túi mình để mở ra ngay trên trình duyệt.
\end{enumerate}

Nếu máy chủ bị tấn công (hacker đột nhập vào kho lưu trữ), chúng chỉ lấy được những chiếc hộp sắt vĩnh viễn không thể mở. Đây chính là giá trị cốt lõi mà AuraCrypt hướng tới.

\section{Các thuật toán mật mã học nền tảng}

\subsection{AES-256-GCM (Advanced Encryption Standard)}
\subsubsection{Lịch sử và bối cảnh ra đời}
Vào những năm 1970, thế giới sử dụng chuẩn mã hóa DES (Data Encryption Standard). Tuy nhiên, đến cuối thập niên 90, sức mạnh máy tính tăng lên nhanh chóng khiến DES (với khóa 56-bit quá ngắn) trở nên lỗi thời và dễ bị bẻ khóa. Năm 1997, Viện Tiêu chuẩn và Công nghệ Quốc gia Hoa Kỳ (NIST) đã tổ chức một cuộc thi toàn cầu để tìm kiếm người kế nhiệm.

\vspace{0.5cm}

Năm 2001, thuật toán \textbf{Rijndael} do hai nhà mật mã học người Bỉ (Joan Daemen và Vincent Rijmen) phát triển đã chiến thắng và trở thành chuẩn AES (Advanced Encryption Standard). Nó được thiết kế để vừa bảo mật cực cao, vừa chạy nhanh trên cả phần cứng hạn chế (như thẻ từ, điện thoại) và siêu máy tính.

\subsubsection{Tại sao AuraCrypt chọn AES-256-GCM?}
AuraCrypt sử dụng phiên bản mạnh nhất là AES-256 kết hợp với chế độ GCM:
\begin{itemize}
    \item \textbf{Độ dài khóa 256-bit:} Hãy hình dung, nếu dùng siêu máy tính mạnh nhất hiện nay để thử từng chìa khóa một (Brute-force), thời gian cần thiết để bẻ khóa AES-256 sẽ lớn hơn cả tuổi thọ của vũ trụ. Điều này đảm bảo dữ liệu an toàn trong nhiều thập kỷ tới.
    \item \textbf{Chế độ GCM (Galois/Counter Mode):} Mã hóa thông thường chỉ giúp che giấu nội dung (tính Bí mật). Nhưng nếu hacker không đọc được, chúng có thể phá hoại bằng cách đổi ngẫu nhiên vài bit trong file mã hóa để làm hỏng dữ liệu gốc khi giải mã. Chế độ GCM giải quyết việc này bằng cách gắn thêm một "con tem niêm phong kỹ thuật số" (Authentication Tag). Khi giải mã, nếu "con tem" bị rách (dữ liệu bị sửa đổi dù chỉ 1 bit), hệ thống sẽ từ chối mở khóa ngay lập tức, đảm bảo tính Toàn vẹn (Integrity).
\end{itemize}

\subsection{PBKDF2 (Password-Based Key Derivation Function 2)}
\subsubsection{Vấn đề của mật khẩu con người}
Con người rất tệ trong việc tạo ra những chuỗi ngẫu nhiên. Chúng ta thường đặt mật khẩu dễ nhớ như "123456", "password@123". Trong khi đó, thuật toán AES cần một khóa 256-bit hoàn toàn ngẫu nhiên và hỗn loạn. Nếu dùng trực tiếp mật khẩu yếu của người dùng làm khóa mã hóa, hacker có thể đoán ra trong tích tắc.

\subsubsection{Giải pháp "kéo giãn" khóa}
PBKDF2 ra đời để làm cầu nối giữa "mật khẩu yếu của con người" và "khóa mạnh cho máy tính". Nó hoạt động theo nguyên lý: \textbf{"Làm chậm kẻ tấn công"}.

\vspace{0.5cm}

Trong AuraCrypt, quy trình này diễn ra như sau:
\begin{itemize}
    \item \textbf{Muối (Salt):} Mỗi người dùng được cấp một chuỗi ký tự ngẫu nhiên duy nhất gọi là "Muối". Muối giúp đảm bảo rằng dù hai người có mật khẩu giống hệt nhau (ví dụ cùng là "123456"), khóa mã hóa tạo ra vẫn sẽ hoàn toàn khác nhau. Điều này chống lại việc hacker dùng một bảng từ điển (Rainbow Table) để tấn công hàng loạt tài khoản cùng lúc.
    \item \textbf{Vòng lặp (Iterations):} AuraCrypt áp dụng 100.000 vòng lặp băm (SHA-256). Để tạo ra khóa, máy tính phải thực hiện phép tính này 100.000 lần. Với người dùng chính chủ, việc này chỉ mất khoảng 0.1 giây (chấp nhận được). Nhưng với hacker muốn thử 1 tỷ mật khẩu/giây, việc bị làm chậm đi 100.000 lần sẽ khiến chi phí tấn công (tiền điện, phần cứng) trở nên quá đắt đỏ và bất khả thi.
\end{itemize}

\section{Các kỹ thuật bảo mật nâng cao}

\subsection{k-Anonymity (Tính ẩn danh trong đám đông)}
Một tính năng quan trọng của AuraCrypt là kiểm tra xem mật khẩu của bạn có bị lộ trong các vụ rò rỉ dữ liệu quá khứ hay không (Password Health Check). Tuy nhiên, làm sao để kiểm tra mà không gửi mật khẩu của bạn cho bên thứ ba (dịch vụ Have I Been Pwned)? Đây là lúc \textbf{k-Anonymity} phát huy tác dụng.

\subsubsection{Nguyên lý hoạt động}
Thay vì gửi mật khẩu thô (ví dụ: "mypassword"), AuraCrypt băm nó thành chuỗi mã hóa SHA-1 (ví dụ: \texttt{5BAA6...}). Chúng tôi chỉ gửi \textbf{5 ký tự đầu tiên} (\texttt{5BAA6}) lên dịch vụ kiểm tra. Dịch vụ này sẽ trả về danh sách hàng trăm mật khẩu khác nhau có cùng 5 ký tự đầu đó.

\vspace{0.5cm}

Trình duyệt của người dùng sau đó sẽ âm thầm kiểm tra trong danh sách tải về xem có cái nào khớp hoàn toàn với phần còn lại của mã băm mật khẩu hay không.
\begin{itemize}
    \item \textbf{Kết quả:} Người dùng biết được mật khẩu mình có an toàn không.
    \item \textbf{Bảo mật:} Dịch vụ kiểm tra không bao giờ biết chính xác người dùng đang dùng mật khẩu nào, vì họ nhận được yêu cầu giống hệt nhau từ hàng ngàn người có cùng 5 ký tự đầu mã băm. Đây là nguyên lý "ẩn mình trong đám đông".
\end{itemize}

\section{Công nghệ phát triển ứng dụng}

\subsection{Web Crypto API: Trái tim bảo mật}
Đây là thành phần quan trọng nhất giúp AuraCrypt thực hiện kiến trúc Zero-Knowledge ngay trên trình duyệt web. Trước đây, việc mã hóa trên JavaScript thường bị coi là không an toàn do hiệu năng kém và dễ bị tấn công qua kênh kề (side-channel attacks). \textbf{Web Crypto API} ra đời để giải quyết vấn đề này.

\subsubsection{Đặc điểm kỹ thuật}
Web Crypto API là một giao diện lập trình tiêu chuẩn W3C, cung cấp các nguyên ngữ mật mã học (cryptographic primitives) được tích hợp sâu trong nhân của trình duyệt (Chrome, Firefox, Edge...). Các thuật toán này không chạy bằng JavaScript mà được thực thi bởi mã máy (thường là C++ hoặc Rust) của trình duyệt, mang lại hai lợi ích lớn:
\begin{itemize}
    \item \textbf{Tốc độ:} Nhanh hơn hàng chục lần so với các thư viện JS thuần túy.
    \item \textbf{An toàn:} Bộ nhớ chứa khóa được quản lý bởi trình duyệt, khó bị truy xuất trái phép hơn. Sử dụng bộ sinh số ngẫu nhiên an toàn mã hóa (CSPRNG) thay vì \texttt{Math.random()} vốn dễ đoán.
\end{itemize}

\subsubsection{Ứng dụng trong AuraCrypt}
AuraCrypt sử dụng triệt để các hàm sau của Web Crypto API:
\begin{itemize}
    \item \texttt{window.crypto.getRandomValues()}: Để sinh ra Vector khởi tạo (IV) ngẫu nhiên 12 bytes cho mỗi lần mã hóa, đảm bảo hai lần mã hóa cùng một dữ liệu sẽ cho ra hai kết quả khác nhau.
    \item \texttt{window.crypto.subtle.importKey()}: Chuyển đổi chuỗi mật khẩu thô của người dùng thành định dạng khóa (KeyMaterial) mà trình duyệt có thể xử lý.
    \item \texttt{window.crypto.subtle.deriveKey()}: Thực thi thuật toán PBKDF2 để dẫn xuất khóa chính thức từ KeyMaterial và Salt.
    \item \texttt{window.crypto.subtle.encrypt()} và \texttt{decrypt()}: Thực thi thuật toán AES-GCM để mã hóa và giải mã dữ liệu JSON.
\end{itemize}

\subsection{React và TypeScript: Nền tảng hiện đại}
AuraCrypt được xây dựng như một ứng dụng Single Page Application (SPA) sử dụng \textbf{React 19}.
\begin{itemize}
    \item \textbf{React (User Interface):} Giúp chia nhỏ giao diện thành các thành phần độc lập (Component) như: Ô nhập liệu, Thẻ hiển thị mật khẩu, Modal... giúp việc phát triển nhanh chóng, mã nguồn dễ bảo trì và trải nghiệm người dùng mượt mà như ứng dụng Desktop.
    \item \textbf{TypeScript:} Là phiên bản nâng cấp của JavaScript với tính năng kiểm tra kiểu dữ liệu nghiêm ngặt. Trong một ứng dụng bảo mật, việc nhầm lẫn giữa một "chuỗi mật khẩu" (string) và một "mảng byte" (ArrayBuffer) có thể gây ra lỗi mã hóa nghiêm trọng. TypeScript giúp ngăn chặn những lỗi này ngay từ lúc viết mã, đảm bảo tính đúng đắn của luồng dữ liệu.
\end{itemize}

\subsection{Supabase: Hạ tầng đám mây}
Supabase được chọn làm Backend cho AuraCrypt vì tính năng \textbf{Row Level Security (RLS)}. RLS cho phép chúng ta lập trình các quy tắc bảo mật ngay tại tầng cơ sở dữ liệu PostgreSQL: "Chỉ người dùng sở hữu dòng dữ liệu này mới có quyền Xóa hoặc Sửa nó". Điều này tạo ra một lớp bảo vệ thứ hai, ngay cả khi tầng ứng dụng (React) có lỗ hổng, dữ liệu trong Database vẫn được bảo vệ bởi chính nó.
\chapter{PHÂN TÍCH VÀ THIẾT KẾ HỆ THỐNG}

\section{Phân tích yêu cầu}
\subsection{Yêu cầu chức năng}
\begin{itemize}
    \item \textbf{Quản lý mật khẩu:} Thêm, sửa, xóa, xem các mục nhập mật khẩu.
    \item \textbf{Mã hóa/Giải mã:} Bảo vệ dữ liệu mật khẩu bằng mã hóa AES-256-GCM.
    \item \textbf{Tạo mật khẩu mạnh:} Tự động tạo mật khẩu ngẫu nhiên theo các tiêu chí (độ dài, ký tự đặc biệt, số...).
    \item \textbf{Quản lý danh mục:} Phân loại mật khẩu theo các danh mục (ví dụ: Công việc, Cá nhân, Ngân hàng).
    \item \textbf{Tìm kiếm:} Tìm kiếm nhanh các mục nhập mật khẩu.
    \item \textbf{Sao lưu và phục hồi:} Tự động sao lưu dữ liệu và cung cấp khả năng phục hồi từ các bản sao lưu.
    \item \textbf{Nhập/Xuất dữ liệu:} Hỗ trợ nhập/xuất dữ liệu từ/ra file JSON hoặc CSV.
    \item \textbf{Tự động khóa:} Tự động khóa ứng dụng sau một khoảng thời gian không hoạt động.
    \item \textbf{Tạo shortcut (Windows):} Tự động tạo shortcut trên desktop khi chạy lần đầu.
\end{itemize}

\subsection{Yêu cầu phi chức năng}
\begin{itemize}
    \item \textbf{Bảo mật:} Dữ liệu phải được bảo vệ chống lại truy cập trái phép và các cuộc tấn công.
    \item \textbf{Hiệu năng:} Ứng dụng phải hoạt động mượt mà, thời gian mã hóa/giải mã và tìm kiếm nhanh chóng.
    \item \textbf{Độ tin cậy:} Dữ liệu không được mất mát hoặc hỏng hóc. Hệ thống sao lưu phải hoạt động ổn định.
    \item \textbf{Khả năng sử dụng:} Giao diện phải trực quan, dễ hiểu và dễ thao tác.
    \item \textbf{Khả năng mở rộng:} Kiến trúc hệ thống nên cho phép dễ dàng thêm các tính năng mới.
\end{itemize}

\section{Thiết kế kiến trúc hệ thống}
AuraCrypt được thiết kế theo kiến trúc phân lớp, tách biệt rõ ràng các thành phần để dễ bảo trì và mở rộng.

\begin{figure}[H]
    \centering
    \includegraphics[width=0.85\textwidth]{images/architecture_diagram.png}
    \caption{Kiến trúc tổng quan của AuraCrypt}
    \label{fig:architecture_diagram}
\end{figure}
\textbf{Các lớp chính:}
\begin{itemize}
    \item \textbf{Presentation Layer (Lớp giao diện):} Xử lý các tương tác với người dùng thông qua Flet UI. Bao gồm các view (login, main vault, settings).
    \item \textbf{Application Layer (Lớp ứng dụng):} Chứa logic điều khiển chính của ứng dụng, tương tác với các lớp dưới để thực hiện các yêu cầu của người dùng.
    \item \textbf{Service Layer (Lớp dịch vụ):} Cung cấp các dịch vụ cốt lõi như quản lý vault, tạo mật khẩu, quản lý danh mục.
    \item \textbf{Infrastructure Layer (Lớp hạ tầng):} Bao gồm các module xử lý mã hóa, đọc/ghi file, sao lưu và các tương tác với hệ điều hành (ví dụ: tạo shortcut).
\end{itemize}

\section{Thiết kế cơ sở dữ liệu (Cấu trúc Vault)}
Dữ liệu mật khẩu được lưu trữ dưới dạng một file JSON đã mã hóa.
\begin{lstlisting}[caption=Cấu trúc dữ liệu vault mã hóa]
{
    "salt": "base64_encoded_salt",
    "nonce": "base64_encoded_nonce",
    "ciphertext": "base64_encoded_encrypted_data"
}
\end{lstlisting}
Trong đó, `ciphertext` chứa một đối tượng JSON đã mã hóa với cấu trúc sau:
\begin{lstlisting}[caption=Cấu trúc dữ liệu nội dung vault (chưa mã hóa)]
{
    "master_key_fingerprint": "hash_of_derived_key",
    "entries": [
        {
            "id": "uuid",
            "service": "Ten dich vu",
            "username": "Ten nguoi dung",
            "password": "Mat khau",
            "url": "URL lien quan",
            "category": "Danh muc",
            "notes": "Ghi chu",
            "created_at": "timestamp",
            "updated_at": "timestamp"
        }
    ],
    "categories": [
        "Danh muc 1",
        "Danh muc 2"
    ],
    "settings": {
        "auto_lock_timeout": 900,
        "clipboard_clear_delay": 15
    }
}
\end{lstlisting}
\subsection{Quản lý dữ liệu sao lưu}
Các bản sao lưu được lưu trong thư mục \texttt{backups} dưới dạng \texttt{vault\_YYYYMMDD\_HHMMSS.bak}.

\section{Thiết kế giao diện người dùng (UI/UX)}
Giao diện người dùng được thiết kế dựa trên nguyên tắc tối giản, hiện đại và dễ sử dụng.
\begin{itemize}
    \item \textbf{Màn hình Đăng nhập/Thiết lập:} Đơn giản, rõ ràng, hướng dẫn người dùng thiết lập mật khẩu chủ lần đầu.
    \item \textbf{Màn hình chính (Vault):} Hiển thị danh sách mật khẩu, thanh tìm kiếm, bộ lọc danh mục và các nút chức năng (thêm, sửa, xóa).
    \item \textbf{Form thêm/sửa mật khẩu:} Các trường nhập liệu rõ ràng, tích hợp nút tạo mật khẩu mạnh.
\end{itemize}

\begin{figure}[H]
    \centering
    \includegraphics[width=0.8\textwidth]{images/mock_login_screen.png}
    \caption{Thiết kế giao diện màn hình Đăng nhập}
    \label{fig:login_screen}
\end{figure}

\begin{figure}[H]
    \centering
    \includegraphics[width=0.8\textwidth]{images/mock_main_screen.png}
    \caption{Thiết kế giao diện màn hình chính}
    \label{fig:main_screen}
\end{figure}
\newpage
\chapter{TRIỂN KHAI VÀ THỬ NGHIỆM}

\section{Môi trường phát triển và công cụ}
\begin{itemize}
    \item \textbf{Hệ điều hành:} Windows 10/11
    \item \textbf{Ngôn ngữ lập trình:} Python 3.10+
    \item \textbf{Framework UI:} Flet 0.20+
    \item \textbf{Môi trường ảo:} `venv`
    \item \textbf{Công cụ đóng gói:} `PyInstaller` (sử dụng thông qua `flet pack`)
    \item \textbf{IDE:} Visual Studio Code
\end{itemize}

\section{Triển khai các chức năng chính}
\subsection{Module mã hóa (`crypto.py`)}
Module này chứa các hàm để thực hiện mã hóa AES-256-GCM, tạo khóa từ mật khẩu chủ bằng PBKDF2, và quản lý salt/nonce.
\lstinputlisting[language=Python, caption=Trích đoạn code mã hóa AES-256-GCM]{core/crypto.py}

\subsection{Module Vault (`vault.py`)}
Module `vault.py` quản lý việc đọc, ghi, mã hóa và giải mã toàn bộ dữ liệu vault. Nó tương tác với `crypto.py` để bảo mật dữ liệu.
\lstinputlisting[language=Python, caption=Trích đoạn code quản lý Vault]{core/vault.py}

\subsection{Quản lý Desktop Shortcut (Windows)}
Tính năng này được triển khai trong `utils/shortcut\_manager.py`, sử dụng thư viện `pywin32` và `winshell` để tương tác với Windows API. Shortcut được tạo tự động khi ứng dụng chạy lần đầu.
\lstinputlisting[language=Python, caption=Trích đoạn code tạo shortcut tự động]{utils/shortcut_manager.py}

\subsection{Giao diện người dùng với Flet}
Flet được sử dụng để xây dựng các thành phần giao diện một cách phản ứng (reactive). Mỗi màn hình (View) được thiết kế là một lớp riêng biệt.
\lstinputlisting[language=Python, caption=Trích đoạn code giao diện màn hình đăng nhập (`ui/login\_view.py`)]{ui/login\_view.py}

\section{Thử nghiệm và đánh giá}
\subsection{Thử nghiệm chức năng}
Chúng em đã thực hiện thử nghiệm chức năng cho từng tính năng của ứng dụng, bao gồm:
\begin{itemize}
    \item Thêm, sửa, xóa, tìm kiếm mật khẩu.
    \item Tạo mật khẩu ngẫu nhiên với các tiêu chí khác nhau.
    \item Đăng nhập/đăng xuất, tự động khóa ứng dụng.
    \item Sao lưu và phục hồi dữ liệu.
    \item Nhập/xuất dữ liệu.
    \item Tạo shortcut tự động (trên Windows).
\end{itemize}
Tất cả các chức năng đều hoạt động đúng như thiết kế.

\subsection{Thử nghiệm bảo mật}
\begin{itemize}
    \item \textbf{Kiểm tra mã hóa:} Xác nhận rằng dữ liệu lưu trữ không thể đọc được nếu không có mật khẩu chủ.
    \item \textbf{Kiểm tra hiệu suất PBKDF2:} Đánh giá thời gian cần thiết để dẫn xuất khóa, đảm bảo độ trễ hợp lý chống brute-force.
    \item \textbf{Kiểm tra làm sạch bộ nhớ đệm:} Đảm bảo mật khẩu trên clipboard được xóa sau thời gian quy định.
\end{itemize}
Kết quả thử nghiệm cho thấy AuraCrypt đạt được các yêu cầu bảo mật đã đề ra.

\subsection{Thử nghiệm đóng gói ứng dụng}
Ứng dụng đã được đóng gói thành công thành tệp `.exe` cho Windows, đảm bảo khả năng chạy độc lập và tạo shortcut tự động với icon chính xác.
\begin{figure}[H]
    \centering
    % TODO \includegraphics[width=0.4\textwidth]{images/desktop_shortcut_final.png}
    \caption{Shortcut AuraCrypt trên desktop Windows}
    \label{fig:desktop_shortcut}
\end{figure}
\newpage
\chapter{KẾT LUẬN VÀ HƯỚNG PHÁT TRIỂN}

\section{Kết luận}
Đồ án "Phần mềm quản lý mật khẩu an toàn AuraCrypt" đã hoàn thành các mục tiêu đề ra. Chúng em đã thành công trong việc thiết kế và triển khai một ứng dụng quản lý mật khẩu an toàn, dễ sử dụng, tích hợp các tính năng bảo mật và tiện ích cần thiết. AuraCrypt cung cấp một giải pháp đáng tin cậy để người dùng bảo vệ thông tin đăng nhập quan trọng của mình.

Những điểm nổi bật của đồ án bao gồm:
\begin{itemize}
    \item Hệ thống mã hóa mạnh mẽ với AES-256-GCM và PBKDF2.
    \item Giao diện người dùng trực quan, thân thiện.
    \item Các tính năng tiện ích như tạo mật khẩu mạnh, quản lý danh mục, sao lưu tự động.
    \item Khả năng đóng gói thành ứng dụng độc lập trên Windows, tự động tạo shortcut.
\end{itemize}
Trong quá trình thực hiện, chúng em đã học hỏi được nhiều kiến thức quý báu về mật mã học, kiến trúc phần mềm và quy trình phát triển ứng dụng.

\section{Hạn chế}
Mặc dù đã đạt được nhiều thành công, đồ án vẫn còn một số hạn chế:
\begin{itemize}
    \item \textbf{Chưa hỗ trợ đa nền tảng hoàn chỉnh:} Hiện tại chỉ tập trung tối ưu cho Windows, chưa thử nghiệm đầy đủ trên macOS hoặc Linux với các tính năng đặc thù (ví dụ: tạo shortcut).
    \item \textbf{Chưa tích hợp tự động điền/đăng nhập:} Người dùng vẫn phải sao chép/dán mật khẩu thủ công.
    \item \textbf{Chưa có tính năng đồng bộ hóa:} Dữ liệu chỉ được lưu trữ cục bộ, chưa có cơ chế đồng bộ hóa an toàn giữa các thiết bị.
    \item \textbf{Giao diện cơ bản:} Mặc dù thân thiện, giao diện vẫn có thể được cải thiện để hiện đại và linh hoạt hơn.
\item \textbf{Xử lý ngoại lệ:} Một số trường hợp ngoại lệ hiếm gặp có thể chưa được xử lý tối ưu.
\end{itemize}

\section{Hướng phát triển trong tương lai}
Để cải thiện và mở rộng AuraCrypt, chúng em đề xuất các hướng phát triển sau:
\begin{itemize}
    \item \textbf{Tích hợp tính năng tự động điền (Autofill):} Phát triển tiện ích mở rộng trình duyệt hoặc tính năng tự động điền vào các trường đăng nhập trên website.
    \item \textbf{Đồng bộ hóa dữ liệu an toàn:} Nghiên cứu và triển khai cơ chế đồng bộ hóa dữ liệu mã hóa giữa các thiết bị thông qua các dịch vụ đám mây (ví dụ: Google Drive, Dropbox) với bảo mật từ đầu đến cuối (end-to-end encryption).
    \item \textbf{Hỗ trợ đa nền tảng tốt hơn:} Tối ưu hóa cho macOS và Linux, bao gồm cả các tính năng đặc thù của từng hệ điều hành.
    \item \textbf{Xác thực đa yếu tố (MFA):} Tích hợp các phương pháp xác thực thứ cấp (ví dụ: sinh trắc học, TOTP) để tăng cường bảo mật.
    \item \textbf{Cải thiện UI/UX:} Nâng cấp giao diện người dùng, bổ sung thêm các tùy chỉnh và chủ đề (themes) để cá nhân hóa trải nghiệm.
    \item \textbf{Kiểm tra bảo mật chuyên sâu:} Thực hiện các cuộc kiểm tra thâm nhập (penetration testing) và đánh giá lỗ hổng bảo mật chuyên nghiệp.
\end{itemize}

Chúng em hy vọng đồ án này sẽ là nền tảng vững chắc cho các nghiên cứu và phát triển tiếp theo trong lĩnh vực quản lý mật khẩu và bảo mật thông tin.
\newpage

% --- PHỤ LỤC (Nếu có) ---
% \appendix % Bắt đầu phần phụ lục
% \input{phuluc_a}
% \input{phuluc_b}

% --- TÀI LIỆU THAM KHẢO ---
% \nocite{*} % Để hiển thị tất cả các mục trong references.bib ngay cả khi không được trích dẫn
% \printbibliography % Nếu dùng biblatex
% Nếu dùng bibtex thủ công:
% \bibliographystyle{plain}
% \bibliography{references}

\end{document}